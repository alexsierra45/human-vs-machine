
\documentclass[12pt,a4paper]{article}
\usepackage[utf8]{inputenc}
\usepackage[spanish]{babel}
\usepackage{amsmath}
\usepackage{graphicx}
\usepackage{hyperref}
\usepackage{multicol}
\usepackage{geometry}
\geometry{margin=1in}

\title{Deep Blue: Hombre vs Máquina}
\author{Juan Carlos Espinosa Delgado, Alex Sierra Alcalá}
\date{\today}

\begin{document}

\maketitle

\begin{abstract}
Este trabajo aborda en profundidad los componentes técnicos y operativos de Deep Blue, la supercomputadora de IBM que marcó un antes y un después en la inteligencia artificial al derrotar al campeón mundial de ajedrez Garry Kasparov en 1997. El documento se estructura en torno a su arquitectura hardware, los algoritmos que definieron su desempeño y el impacto trascendental en la ciencia de la computación. Además, se exploran aspectos biográficos de los desarrolladores clave y cómo sus contribuciones específicas permitieron alcanzar este hito. El trabajo incluye análisis detallados, ejemplos, gráficos y referencias técnicas para ofrecer una comprensión integral del tema.
\end{abstract}

\section*{Palabras clave}
Deep Blue, supercomputación, inteligencia artificial, ajedrez, computación de alto rendimiento.

\section{Introducción}
En 1997, la supercomputadora Deep Blue sorprendió al mundo al derrotar al entonces campeón mundial de ajedrez, Garry Kasparov, en un evento que simbolizó el avance tecnológico en inteligencia artificial y supercomputación. Deep Blue no solo fue una máquina diseñada para jugar ajedrez, sino un ejemplo paradigmático de cómo integrar hardware especializado y algoritmos avanzados para resolver problemas complejos. 

A lo largo de este documento, se explorará cómo Deep Blue logró un rendimiento sin precedentes al analizar hasta 200 millones de posiciones por segundo, una capacidad que superaba con creces cualquier esfuerzo computacional previo en ajedrez. El trabajo examina sus componentes hardware, el diseño del software y los algoritmos que le permitieron tomar decisiones estratégicas comparables a las de un gran maestro humano. También se discutirá el impacto histórico y técnico de este logro, que sentó las bases para desarrollos futuros en inteligencia artificial.

\subsection{Contexto histórico}

El camino hacia la creación de Deep Blue se enmarca dentro de un desafío histórico en la computación: desarrollar una máquina capaz de competir al nivel de un campeón mundial de ajedrez. Este objetivo, denominado un \textit{Grand Challenge} de la informática, tuvo sus primeros avances conceptuales con Charles Babbage en la década de 1840, quien contempló la posibilidad de jugar ajedrez con su Máquina Analítica. Más tarde, Claude Shannon \cite{Shannon01031950} y Alan Turing, en la década de 1940, sentaron las bases teóricas del ajedrez computarizado.

\textbf{Progresos iniciales (1950-1970):}  
\begin{itemize}
    \item En la década de 1950, Herbert Simon predijo que una computadora sería campeona mundial de ajedrez antes de 1967 \cite{simonherbertspeech}. Sin embargo, los avances iniciales eran modestos: en 1966, las computadoras solo alcanzaban niveles de jugador principiante.
    \item En 1977, el programa \textit{Chess 4.5} \cite{slate1977chess} logró jugar al nivel de Clase A (aficionado fuerte) gracias a los avances en hardware, como la supercomputadora Cyber 176.
\end{itemize}

\textbf{El surgimiento de hardware especializado (1970-1980):}  
\begin{itemize}
    \item Ken Thompson \cite{condon1982belle} de Bell Labs desarrolló Belle, una máquina especial con circuitos dedicados que, en 1982, fue la primera computadora en alcanzar el nivel de Maestro Nacional. Esto marcó un hito en el uso de hardware personalizado, como circuitos VLSI para la generación de movimientos.
\end{itemize}

\textbf{El impacto de la búsqueda selectiva y hardware más rápido (1980-1990):}  
\begin{itemize}
    \item Durante los años 80, proyectos como Cray Blitz \cite{hyatt1985parallel} y Hitech utilizaron estrategias avanzadas de búsqueda y procesadores especializados. Hitech, desarrollado en la Universidad Carnegie Mellon, empleó un diseño con 64 chips, uno para cada casilla del tablero, logrando niveles de juego de alto rendimiento \cite{ebeling1987all}.
    \item Feng-hsiung Hsu, en 1985, refinó el diseño de Belle para desarrollar Deep Thought, un precursor directo de Deep Blue\cite{hsu1990deep, hsu1987two, hsu1989chess}. En 1988, Deep Thought alcanzó el nivel de Gran Maestro y ganó el Premio Fredkin por ser la primera máquina en lograrlo en partidas oficiales.
\end{itemize}

\subsection{Aspectos Biográficos Relevantes}
El éxito de Deep Blue no habría sido posible sin las contribuciones de un equipo multidisciplinario de expertos. A continuación, se detalla el rol de dos figuras clave:

\subsubsection*{Feng-hsiung Hsu}
Feng-hsiung Hsu, nacido en Taiwán en 1959, es reconocido como el arquitecto principal de Deep Blue \cite{hsu2002deep}. Desde una temprana edad, Hsu mostró un gran interés por la ingeniería y la computación, lo que lo llevó a estudiar ingeniería eléctrica en Taiwán antes de trasladarse a Estados Unidos para realizar su doctorado en la Universidad Carnegie Mellon. Allí, bajo la supervisión de destacados investigadores en inteligencia artificial, desarrolló el proyecto Deep Thought, el precursor directo de Deep Blue.

Aportaciones técnicas clave:
Hsu se especializó en el diseño de chips VLSI (Very Large Scale Integration), que resultaron esenciales para optimizar las capacidades computacionales de las máquinas de ajedrez. Estos chips permitieron a Deep Thought, y más tarde a Deep Blue, analizar millones de posiciones por segundo, revolucionando el enfoque tradicional de la programación de ajedrez, que hasta entonces dependía principalmente de software y hardware generalistas.

En IBM:
En 1989, Hsu se unió al Centro de Investigación T.J. Watson de IBM, donde lideró el desarrollo de hardware personalizado para Deep Blue. Su trabajo se centró en diseñar un sistema especializado que maximizara la velocidad de análisis y minimizara los cuellos de botella en la evaluación de posiciones de ajedrez.
Hsu no solo aportó soluciones técnicas innovadoras, sino que también defendió un enfoque integral en el que el hardware y el software trabajaran en sinergia. Este enfoque fue clave para superar los desafíos de búsqueda y evaluación posicional que enfrentaba Deep Blue.

Legado y contribución:
El impacto de Hsu va más allá de Deep Blue. Su trabajo inspiró a una generación de investigadores en inteligencia artificial y computación de alto rendimiento, demostrando cómo un diseño especializado puede superar las limitaciones de los sistemas generalistas.

\subsubsection*{Murray Campbell y el equipo de IBM}
Murray Campbell, nacido en Canadá, fue otro de los pilares fundamentales del equipo que desarrolló Deep Blue. Con un doctorado en inteligencia artificial, Campbell se especializó en el desarrollo de algoritmos de búsqueda y evaluación estratégica en juegos como el ajedrez.

Rol en el proyecto:
Campbell trabajó estrechamente con Hsu para integrar el hardware especializado de Deep Blue con un software optimizado que aprovechara plenamente las capacidades de procesamiento del sistema. Su enfoque principal fue el diseño y la implementación de estrategias que permitieran a Deep Blue tomar decisiones de manera eficiente incluso en situaciones altamente complejas.

Equipo multidisciplinario:
El equipo que construyó Deep Blue en IBM no estaba compuesto únicamente por ingenieros y programadores; también incluía expertos en ajedrez, científicos computacionales y técnicos especializados en sistemas de alto rendimiento. Esta colaboración permitió abordar los desafíos del proyecto desde múltiples perspectivas, resolviendo problemas técnicos y estratégicos de manera innovadora.
Entre los logros del equipo destaca la creación de una base de datos de aperturas y finales, así como funciones heurísticas avanzadas que dotaron a Deep Blue de un entendimiento posicional comparable al de los mejores jugadores humanos.

Contribución de Campbell al software de búsqueda:
Campbell lideró el desarrollo del algoritmo de búsqueda Alpha-Beta optimizado que utilizaba Deep Blue. Este algoritmo no solo permitió explorar movimientos más prometedores, sino que también garantizó una velocidad de respuesta acorde con las necesidades de un enfrentamiento en tiempo real contra oponentes humanos.

Impacto global:
El trabajo de Campbell y el equipo de IBM mostró cómo la combinación de hardware especializado y estrategias avanzadas de software podía resolver problemas que antes se consideraban insuperables. Esta sinergia entre disciplinas se convirtió en un modelo para proyectos posteriores en inteligencia artificial y computación de alto rendimiento.

\subsection{Precedentes}

Antes de la creación de Deep Blue, existieron varios desarrollos clave que allanaron el camino para esta supercomputadora.

\textbf{ChipTest y Deep Thought:} En los años 80, en la Universidad Carnegie Mellon, se desarrollaron ChipTest y Deep Thought como pasos iniciales hacia la creación de una máquina de ajedrez competitiva. En 1988, Deep Thought se convirtió en la primera computadora en derrotar a un Gran Maestro en un torneo, logrando velocidades de búsqueda de entre 500,000 y 700,000 posiciones por segundo, gracias a un chip especializado en generación de movimientos.

\textbf{Deep Thought 2:} Entre 1989 y 1990, el equipo de Deep Thought (Anantharaman, Campbell y Hsu) se trasladó al Centro de Investigación T.J. Watson de IBM para desarrollar una máquina de ajedrez de clase mundial. Deep Thought 2, considerada un prototipo de Deep Blue, se diseñó como un paso intermedio y participó en eventos públicos entre 1991 y 1995. Sus principales mejoras incluyeron:  
\begin{itemize}
    \item \textbf{Multiprocesamiento a escala media:} Se equipó inicialmente con 24 motores de ajedrez, en comparación con los 2 de su predecesor.
    \item \textbf{Hardware de evaluación mejorado:} Incorporó RAMs más grandes y características adicionales en su función de evaluación, aunque con limitaciones como la incapacidad de reconocer finales complejos, lo que se resolvió parcialmente con soluciones de software.
    \item \textbf{Software de búsqueda mejorado:} El software fue reescrito para optimizar la búsqueda paralela y ampliar extensiones de búsqueda, formando la base del software de Deep Blue.
    \item \textbf{Libro extendido:} Permitió movimientos iniciales razonables incluso sin un libro de aperturas formal, característica heredada por Deep Blue.
\end{itemize}

Deep Thought 2 logró victorias importantes, como los campeonatos de ajedrez informático ACM en 1991 y 1994, y un triunfo 3-1 contra el equipo nacional de Dinamarca en 1993.

\section{Componentes Técnicos de Deep Blue}

\subsection{Arquitectura Hardware}
Deep Blue fue diseñado como un sistema altamente especializado para ajedrez, lo que le permitió obtener un rendimiento incomparable en su época. Los componentes principales incluyen:

\subsubsection*{Procesadores de propósito general}
Deep Blue utilizó nodos IBM RS/6000, una arquitectura de alto rendimiento basada en RISC (Reduced Instruction Set Computing). Estos procesadores manejaban tareas generales, incluyendo la coordinación de los cálculos paralelos realizados por los chips especializados.

\subsubsection*{Chips VLSI personalizados}
El corazón del rendimiento de Deep Blue residía en sus chips VLSI (Very Large Scale Integration), diseñados específicamente para realizar cálculos relacionados con el ajedrez. Cada chip podía analizar hasta 2 millones de posiciones por segundo, lo que permitía a la máquina realizar una búsqueda exhaustiva en el árbol de posibilidades del juego.

\subsubsection*{Sistema paralelo masivo}
El sistema estaba compuesto por 30 nodos de procesamiento, cada uno equipado con 480 chips personalizados. Esta configuración permitía analizar movimientos de manera simultánea y eficiente. La arquitectura paralela fue clave para alcanzar la capacidad de 200 millones de posiciones por segundo.

\subsection{Software y Algoritmos}
El software de Deep Blue era igualmente avanzado y estaba diseñado para maximizar el uso del hardware especializado:

\subsubsection*{Algoritmo de búsqueda Alpha-Beta}
El algoritmo Alpha-Beta es una optimización de la búsqueda minimax que permite reducir drásticamente el número de nodos evaluados. En Deep Blue, se implementó una versión mejorada que priorizaba las ramas más prometedoras, logrando decisiones rápidas y precisas.

\subsubsection*{Función de evaluación heurística}
La función de evaluación de Deep Blue consideraba múltiples factores, incluyendo:
\begin{itemize}
    \item \textbf{Control del tablero:} Evaluaba la influencia de las piezas sobre áreas clave.
    \item \textbf{Ventaja material:} Calculaba el equilibrio de fuerzas en términos de puntos de piezas.
    \item \textbf{Estrategias tácticas:} Identificaba patrones como sacrificios y jaques.
\end{itemize}

\subsubsection*{Base de datos de aperturas y finales}
La base de datos de Deep Blue incluía millones de posiciones precomputadas, lo que le proporcionaba una ventaja inicial significativa en las partidas y un conocimiento profundo en finales complejos.

\section{Impacto en la Ciencia de la Computación}
El éxito de Deep Blue no solo fue un triunfo técnico, sino que también marcó un cambio de paradigma en la inteligencia artificial y la supercomputación:

\subsubsection*{Supercomputación aplicada}
Deep Blue demostró cómo los sistemas paralelos masivos pueden resolver problemas altamente complejos en tiempos razonables. Este enfoque se ha aplicado desde entonces en áreas como modelado climático y simulaciones médicas.

\subsubsection*{Avances en inteligencia artificial}
Aunque Deep Blue no utilizaba aprendizaje automático, inspiró investigaciones en aprendizaje supervisado y redes neuronales, marcando el inicio de una nueva era en inteligencia artificial.

\subsubsection*{Legado técnico y cultural}
El enfrentamiento entre Deep Blue y Kasparov simbolizó la creciente capacidad de las máquinas para igualar e incluso superar habilidades humanas específicas. Esto planteó preguntas éticas y filosóficas sobre el papel de la inteligencia artificial en la sociedad.

\section{Conclusión}
Deep Blue fue más que una máquina de ajedrez; representó un avance técnico y conceptual que redefinió los límites de la computación. Su éxito fue el resultado de la integración precisa de hardware especializado, software optimizado y el conocimiento de expertos humanos. Este logro sigue siendo un punto de referencia en el desarrollo de tecnologías avanzadas.

\bibliographystyle{plain} % Estilo de bibliografía
\bibliography{references} % Archivo de bibliografía sin la extensión

\end{document}
